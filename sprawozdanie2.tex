\documentclass[11pt]{article}
\usepackage[T1]{fontenc}
\usepackage[polish]{babel}
\usepackage[utf8]{inputenc}
\usepackage{lmodern}
\usepackage{amsmath}
\usepackage{indentfirst}
\usepackage{enumitem}
\usepackage{amsmath}
\usepackage{siunitx}
\usepackage{graphicx}
\usepackage{mathtools}
\usepackage{xcolor}
\usepackage{multicol}
\usepackage{booktabs}
\usepackage{multirow}
\usepackage{colortbl}
\sisetup{
per-mode=fraction
}
\usepackage{geometry}
 \geometry{
 a4paper,
 total={170mm,257mm},
 left=25mm,
 right=25mm,
 top=25mm,
 bottom=25mm,
 }
\begin{document}
\begin{titlepage}
\centering
\includegraphics[width=0.20\textwidth]{RCXD.png}\par\vspace{0.5cm}
	{\scshape\LARGE POLITECHNIKA WROCŁAWSKA \par}
	{\scshape\LARGE WYDZIAŁ CHEMICZNY \par}
	\vspace{0.2 cm}
	\hrulefill
	\\
	\vspace{0.2 cm}
	{\scshape\Large Laboratorium procesów reaktorowych\par}
	\vspace{2.5cm}
	{\huge\bfseries Sprawozdanie\par}
	\vspace{0.5cm}
	{\huge 6. Ekstrakcja z reakcją chemiczną w przepływowym reaktorze zbiornikowym\par}
	\vspace{2.5cm}
	{\scshape\Large\bf GRUPA I\par}
	\vspace{0.25cm}
	{\Large\itshape Maciej Wieruszyński\\ Klaudia Jędrzejek\\Julia Ołdak
	\par}
	\vspace{1cm}
	{\scshape\Large piątek 8:30-11:15\par}
	\vfill	
	{\Large \today\par}
\end{titlepage}
\section{Cel ćwiczenia:}
\begin{enumerate}[label=(\alph*)]
\item Wyznaczenie współczynnika podziału bezwodnika kwasu octowego w układzie 
toluen - bezwodnik kwasu octowego - woda.
\item Wyznaczenie  objętościowego  współczynnika   przenikania   masy   i   porównanie 
wartości eksperymentalnej z obliczaną ze znanych z literatury korelacji.
\item Obliczenia sprawności stopnia na podstawie uzyskanych wyników.
\end{enumerate}
\section{Aparatura:}
\begin{figure}[h!]
\centering
\includegraphics[scale=0.1165]{aparatura.jpg}
\caption{Aparatura procesowa}
\end{figure}
\begin{multicols}{3}
\begin{enumerate}
\item Konduktometr
\item Pompa perystalyczna do tłoczenia bezwodnika
\item Pompa perystalyczna do tłoczenia wody destylowanej
\item Termopara
\item Elektroda do pomiaru konduktywności
\item Mieszadło
\item Reaktor przepływowy
\item Zbiorniczek przelewowy
\item Wylot produktu
\item Wlot składnika
\item Zbiornik wody
\item Wlot składnika
\item Zbiornik bezwodnika
\item Zbiornik produktu
\item Termometr kontrolujący termostat
\item Pokrętło zmiany temperatury
\item Termostatujący wymiennik ciepła
\end{enumerate}
\end{multicols}
\section{Metodyka pomiarów:}
\par Eksperymenty prowadzi się na układzie toluen - bezwodnik kwasu octowego - 
woda. Substrat A (bezwodnik) jest doprowadzany do reaktora z faza organiczną, z 
której dyfunduje do fazy wodnej, gdzie zachodzi hydroliza do kwasu octowego wg. 
Schematu $A \rightarrow 2B$.
\par W celu wykonania pomiarów kinetycznych należy jedną butlę napełnić wodą 
destylowaną, drugą zaś roztworem bezwodnika kwasu octowego w toluenie o stężeniu 
$C_{{AR}_0}$ , które podaje prowadzący.  Następnie włącza się termostat i pompki  dozujące 
ustawiając na żądany wydatek $\dot{V}_E$ i $\dot{V}_R$.
\section{Zestawienie wyników pomiarów:}
% Table generated by Excel2LaTeX from sheet 'Arkusz1'
\begin{table}[htbp]
  \centering
  \caption{Wyniki pomiarów}
    \begin{tabular}{rrrrrrrrrrr}
    \toprule
    \multicolumn{1}{c}{SUROWIEC} & \multicolumn{2}{c}{FAZA ORG.} & \multicolumn{3}{c}{FAZA WODNA} & \multicolumn{1}{c}{$V$} & \multicolumn{1}{c}{$V_E$} & \multicolumn{1}{c}{$V_R$} & \multicolumn{1}{c}{$\dot{V}_E$} & \multicolumn{1}{c}{$\dot{V}_R$} \\
    \midrule
    \multicolumn{1}{l}{$V_{NaOH}$ $\displaystyle{\left[\SI{}{\cubic\centi\metre}\right]}$} & \multicolumn{2}{c}{$V_{NaOH}$ $\displaystyle{\left[\SI{}{\cubic\centi\metre}\right]}$} & \multicolumn{2}{c}{$V_{NaOH}$ $\displaystyle{\left[\SI{}{\cubic\centi\metre}\right]}$} & \multicolumn{1}{l}{$\sigma$ $\displaystyle{\left[\SI{}{\milli\siemens}\right]}$} & \multicolumn{3}{c}{$\displaystyle{\left[\SI{}{\cubic\centi\metre}\right]}$} & \multicolumn{2}{c}{$\displaystyle{\left[\SI{}{\cubic\metre\per\second}\right]}$} \\
    \midrule
    \multirow{2}[4]{*}{14,7} & 8,7   & \multirow{2}[4]{*}{8,7} & 6,7   & \multirow{2}[4]{*}{6,6} & \multirow{2}[4]{*}{0,45} & \multirow{2}[4]{*}{5,0} & \multirow{2}[4]{*}{5,0} & \multirow{2}[4]{*}{5,0} & \multirow{2}[4]{*}{$\num{2,5E-07}$} & \multirow{2}[4]{*}{$\num{2,5E-07}$} \\
\cmidrule{2-2}\cmidrule{4-4}          & 8,6   &       & 6,5   &       &       &       &       &       &       &  \\
    \bottomrule
    \end{tabular}%
  \label{tab:addlabel}%
\end{table}%
\section{Opracowanie wyników pomiarów:}
\subsection{Wyznaczenie współczynnika podziału bezwodnika kwasu octowego}
\subsubsection{Wyznaczenie stosunku faz wewnątrz reaktora:}
\begin{equation}
\phi=\frac{\dot{V}_R}{\dot{V}_E}=\frac{\SI{2,5E-07}{\cubic\metre\per\second}}{\SI{2,5E-07}{\cubic\metre\per\second}}=1
\end{equation}
\subsubsection{Wyznaczenie stężenia bezwodnika kwasu octowego w surowcu:}
\begin{equation}
C_{{AR}_0}=\frac{V_{NaOH}\times C_{NaOH}}{V_{{R}_0}}=\frac{\SI{14.7}{\cubic\centi\metre}\times \SI{0.1}{\mole\per\cubic\deci\metre}}{\SI{5.0}{\cubic\centi\metre}}=\SI{0.294}{\mole\per\cubic\deci\metre}
\end{equation}
\subsubsection{Wyznaczenie stężenia bezwodnika kwasu octowego w fazie rafinatowej:}
\begin{equation}
C_{AR}=\frac{V_{NaOH}\times C_{NaOH}}{V_{{R}}}=\frac{\SI{8.7}{\cubic\centi\metre}\times \SI{0.1}{\mole\per\cubic\deci\metre}}{\SI{5.0}{\cubic\centi\metre}}=\SI{0.174}{\mole\per\cubic\deci\metre}
\end{equation}
\subsubsection{Wyznaczenie stężenia kwasu octowego w fazie ekstraktowej:}
\begin{equation}
\begin{split}
C_{BE}&=0.1065\sigma^3+0.0858\sigma^2+0.04000\sigma+0.0224\\
C_{BE}&=0.1065\times(\SI{0.45}{\milli\siemens})^3+0.0858\times(\SI{0.45}{\milli\siemens})^2+0.04000\times \SI{0.45}{\milli\siemens}+0.0224\\
C_{BE}&=\SI{0.0675}{\mole\per\cubic\deci\metre}
\end{split}
\end{equation}
\subsubsection{Wyznaczenie stężenia nieprzereagowanego bezwodnika w fazie ekstraktowej:}
\begin{equation}
\begin{split}
C_{AE}&=\frac{\frac{V_{NaOH}\times C_{NaOH}}{V_E}-C_{BE}}{2}\\
C_{AE}&=\frac{\frac{\SI{6.6}{\cubic\centi\metre}\times\SI{0.1}{\mole\per\cubic\deci\metre}}{\SI{5.0}{\cubic\centi\metre}}-\SI{0.0675}{\mole\per\cubic\deci\metre}}{2}\\
C_{AE}&=\SI{0.0323}{\mole\per\cubic\deci\metre}
\end{split}
\end{equation}
\subsubsection{Wyznaczenie stężenia reagentów odniesionej do całkowitej objętości czynnej reaktora:}
\begin{align}
{C'}_{AE}=C_{AE}\times\frac{V_E}{V_C}=C_{AE}\times\frac{1}{2}=C_{BE}=\SI{0.0323}{\mole\per\cubic\deci\metre}\times\frac{1}{2}&=\SI{0.0161}{\mole\per\cubic\deci\metre}\\
{C'}_{AR}=C_{AR}\times\frac{V_E}{V_C}=C_{AE}\times\frac{1}{2}=C_{BE}=\SI{0.0675}{\mole\per\cubic\deci\metre}\times\frac{1}{2}&=\SI{0.0337}{\mole\per\cubic\deci\metre}\\
{C'}_A={C'}_{AE}+{C'}_{AR}=\SI{0.0161}{\mole\per\cubic\deci\metre}+\SI{0.0337}{\mole\per\cubic\deci\metre}&=\SI{0.0499}{\mole\per\cubic\deci\metre}
\end{align}
\subsubsection{Wyznaczenie czasu przestrzennego:}
papapa
\end{document}
